\section[Low-level data types]{\LARGE\underline{Low-level data types}}

    \vspace{10pt}

    This section is dedicated to explain the various proposed low-level data types including integer and floating point. The smallest addressable object in FabRISC is the ``byte'', that is, eight consecutive bits. Longer types are constructed from multiple bytes side by side following powers of two: one, two, four or eight bytes in little-endian order.

    \subsection{Integer types}

        Integers are arguably the most common data types. They are always signed using 2's complement notations and depending on their length, they can have various names:

        \begin{itemize}

            \item \textit{\textbf{Byte}: 8 consecutive bits.}
            \item \textit{\textbf{Short}: 16 consecutive bits, or alternatively, 2 bytes.}
            \item \textit{\textbf{Word}: 32 consecutive bits, or alternatively, 4 bytes.}
            \item \textit{\textbf{Long}: 64 consecutive bits, or alternatively, 8 bytes.}

        \end{itemize}

        Integer types are manipulated by integer instructions which, by default, behave in a modular fashion. Edge cases, such as wraps-around or overflows can be happen in particular situations:

        \begin{itemize}

            \item \textit{\textbf{Carry over}: this situation arises when the absolute value of the result is too big to fit in the desired data type.}

            \item \textit{\textbf{Carry under}: this situation arises when the absolute value of the result is too small to fit in the desired data type.}

            \item \textit{\textbf{Overflow}: this situation arises when the signed value of the result is too big to fit in the desired data type.}

            \item \textit{\textbf{Underflow}: this situation arises when the signed value of the result is too small to fit in the desired data type.}

        \end{itemize}

        % pointer operations

    \subsection{Floating point types}

        [coming soon...]

        % 

    \par\noindent\rule{\textwidth}{0.4pt}
    \textit{[coming soon...]}
    \par\noindent\rule{\textwidth}{0.4pt}