\section[Instruction list]{\LARGE\underline{Instruction list}} %7

    \vspace{10pt}

    This section is dedicated to a full and extensive list of all the proposed instructions in the ISA which are divided into their corresponding macro module and micro module. Each micro module will list its instructions in alphabetical order.

    \subsection{Basic scalar integer computational}

        \vspace{10pt}

        This subsection is dedicated to simple scalar integer computational instructions. This module has the code name ``BSIC`` (basic scalar integer computational) and includes simple arithmetic and logic integer operations.

        \subsubsection{(ADD) Addition}

            This instruction computes \(ra = rb + rc\). ADD is unprivileged; has an opcode of xxx and belongs to the A format. Updated arithmetic flags are the following: COVR, CUND, OVFL, UNFL, ZERO and SIGN. The remaining arithmetic flags are reset to zero. The applied modifier class is 1.

        \par\noindent\rule{\textwidth}{0.4pt}

        \subsubsection{(SUB) Subtraction}

            This instruction computes \(ra = rb - rc\). SUB is unprivileged; has an opcode of xxx and belongs to the A format. Updated arithmetic flags are the following: COVR, CUND, OVFL, UNFL, ZERO and SIGN. The remaining arithmetic flags are reset to zero. The applied modifier class is 1.

        \par\noindent\rule{\textwidth}{0.4pt}

        [to be continued...]

    \subsection{Data transfer instructions}

        [Coming soon...]

    \subsection{Control transfer instructions}

        [Coming soon...]

    \subsection{System instructions}

        [Coming soon...]