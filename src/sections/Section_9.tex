\section[Instruction list]{\LARGE\underline{Instruction list}}

    \vspace{10pt}

    This section is dedicated to a full and extensive list of all the proposed instructions in the ISA which are divided into their corresponding module. Computational instructions can generate flags but, since there is no flags register, they will be ignored unless the AEXC bit is set for automatic arithmetic exceptions.

    \subsection{Computational-integer-scalar-basic}

        \begin{multicols}{2}

            \begin{itemize}

                \item \textit{\textbf{(ADD) Addition}: this instruction computes \((ra = rb + rc)\); belongs to the B format; uses the modifier class 1; has an opcode of xxxxxxxxxxxx and can generate the COVR, OVFL and UNFL flags. This instruction is scalar.}

                \item \textit{\textbf{(SUB) Subtraction}: this instruction computes \((ra = rb - rc)\); belongs to the B format; uses the modifier class 1; has an opcode of xxxxxxxxxxxx and can generate the COVR, OVFL and UNFL flags. This instruction is scalar.}

                \item \textit{\textbf{(AND) Bitwise AND}: this instruction computes \((ra = rb \wedge rc)\); belongs to the B format; uses the modifier class 1; has an opcode of xxxxxxxxxxxx and doesn't generate any flags. This instruction is scalar.}

                \item \textit{\textbf{(NAND) Bitwise NAND}: this instruction computes \((ra = \neg(rb \wedge rc))\); belongs to the B format; uses the modifier class 1; has an opcode of xxxxxxxxxxxx and doesn't generate any flags. This instruction is scalar.}

                \item \textit{\textbf{(OR) Bitwise OR}: this instruction computes \((ra = rb \vee rc)\); belongs to the B format; uses the modifier class 1; has an opcode of xxxxxxxxxxxx and doesn't generate any flags. This instruction is scalar.}

                \item \textit{\textbf{(NOR) Bitwise NOR}: this instruction computes \((ra = \neg(rb \vee rc))\); belongs to the B format; uses the modifier class 1; has an opcode of xxxxxxxxxxxx and doesn't generate any flags. This instruction is scalar.}

                \item \textit{\textbf{(XOR) Bitwise XOR}: this instruction computes \((ra = rb \oplus rc)\); belongs to the B format; uses the modifier class 1; has an opcode of xxxxxxxxxxxx and doesn't generate any flags. This instruction is scalar.}

                \item \textit{\textbf{(XNOR) Bitwise XNOR}: this instruction computes \((ra = \neg(rb \oplus rc))\); belongs to the B format; uses the modifier class 1; has an opcode of xxxxxxxxxxxx and doesn't generate any flags. This instruction is scalar.}

                \item \textit{\textbf{(ALSH) Arithmetic left shift}: this instruction computes \((ra = rb <<< rc)\); belongs to the B format; uses the modifier class 1; has an opcode of xxxxxxxxxxxx and can generate the xxx flags. This instruction is scalar.}

                \item \textit{\textbf{(ARSH) Arithmetic right shift}: this instruction computes \((ra = rb >>> rc)\); belongs to the E format; uses the modifier class 1; has an opcode of xxxxxxxxxxxx and can generate the xxx flags. This instruction is scalar.}

                \item \textit{\textbf{(LLSH) Logical left shift}: this instruction computes \((ra = rb << rc)\); belongs to the E format; uses the modifier class 1; has an opcode of xxxxxxxxxxxx and can generate the xxx flags. This instruction is scalar.}

                \item \textit{\textbf{(LRSH) Logical right shift}: this instruction computes \((ra = rb >> rc)\); belongs to the E format; uses the modifier class 1; has an opcode of xxxxxxxxxxxx and can generate the xxx flags. This instruction is scalar.}

                \item \textit{\textbf{(IADD) Immediate addition}: this instruction computes \((ra = rb + signxt(imm_8))\); belongs to the E format; uses the modifier class 1; has an opcode of xxxxxxxxxxxx and can generate the xxx flags. This instruction is scalar.}

                \item \textit{\textbf{(ISUB) Immediate subtraction}: this instruction computes \((ra = rb - signxt(imm_8))\); belongs to the E format; uses the modifier class 1; has an opcode of xxxxxxxxxxxx and can generate the xxx flags. This instruction is scalar.}

                \item \textit{\textbf{(IAND) Immediate bitwise AND}: this instruction computes \((ra = rb \wedge zeroxt(imm_8))\); belongs to the E format; uses the modifier class 1; has an opcode of xxxxxxxxxxxx and can generate the xxx flags. This instruction is scalar.}

                \item \textit{\textbf{(INAND) Immediate bitwise NAND}: this instruction computes \((ra = \neg(rb \wedge zeroxt(imm_8)))\); belongs to the E format; uses the modifier class 1; has an opcode of xxxxxxxxxxxx and can generate the xxx flags. This instruction is scalar.}

                \item \textit{\textbf{(IOR) Immediate bitwise OR}: this instruction computes \((ra = rb \vee zeroxt(imm_8))\); belongs to the E format; uses the modifier class 1; has an opcode of xxxxxxxxxxxx and can generate the xxx flags. This instruction is scalar.}

                \item \textit{\textbf{(INOR) Immediate bitwise NOR}: this instruction computes \((ra = \neg(rb \vee zeroxt(imm_8)))\); belongs to the E format; uses the modifier class 1; has an opcode of xxxxxxxxxxxx and can generate the xxx flags. This instruction is scalar.}

                \item \textit{\textbf{(IXOR) Immediate bitwise XOR}: this instruction computes \((ra = rb \oplus zeroxt(imm_8))\); belongs to the E format; uses the modifier class 1; has an opcode of xxxxxxxxxxxx and can generate the xxx flags. This instruction is scalar.}

                \item \textit{\textbf{(IXNOR) Immediate bitwise XNOR}: this instruction computes \((ra = \neg(rb \oplus zeroxt(imm_8)))\); belongs to the E format; uses the modifier class 1; has an opcode of xxxxxxxxxxxx and can generate the xxx flags. This instruction is scalar.}

                \item \textit{\textbf{(IASH) Immediate arithmetic shift}: this instruction computes \((ra = rb <<<>>> imm_8)\); belongs to the E format; uses the modifier class 1; has an opcode of xxxxxxxxxxxx and can generate the xxx flags. This instruction is scalar and the direction of the shift is dictated by imm[7] bit.}

                \item \textit{\textbf{(ILSH) Immediate logical shift}: this instruction computes \((ra = rb <<>> imm_8)\); belongs to the E format; uses the modifier class 1; has an opcode of xxxxxxxxxxxx and can generate the xxx flags. This instruction is scalar and the direction of the shift is dictated by imm[7] bit.}

                \item \textit{\textbf{(LIADD) Long immediate addition}: this instruction computes \((ra = rb + signxt(imm_{26}))\); belongs to the E.l format; uses the modifier class 1; has an opcode of xxxxxxxxxxxx and can generate the xxx flags. This instruction is scalar.}

                \item \textit{\textbf{(LISUB) Long immediate subtraction}: this instruction computes \((ra = rb - signxt(imm_{26}))\); belongs to the E.l format; uses the modifier class 1; has an opcode of xxxxxxxxxxxx and can generate the xxx flags. This instruction is scalar.}

                \item \textit{\textbf{(LIAND) Long Immediate bitwise AND}: this instruction computes \((ra = rb \wedge zeroxt(imm_{26}))\); belongs to the E format; uses the modifier class 1; has an opcode of xxxxxxxxxxxx and can generate the xxx flags. This instruction is scalar.}

                \item \textit{\textbf{(LINAND) Long Immediate bitwise NAND}: this instruction computes \((ra = \neg(rb \wedge zeroxt(imm_{26})))\); belongs to the E.l format; uses the modifier class 1; has an opcode of xxxxxxxxxxxx and can generate the xxx flags. This instruction is scalar.}

                \item \textit{\textbf{(LIOR) Long Immediate bitwise OR}: this instruction computes \((ra = rb \vee zeroxt(imm_{26}))\); belongs to the E.l format; uses the modifier class 1; has an opcode of xxxxxxxxxxxx and can generate the xxx flags. This instruction is scalar.}

                \item \textit{\textbf{(LINOR) Long Immediate bitwise NOR}: this instruction computes \((ra = \neg(rb \vee zeroxt(imm_{26})))\); belongs to the E.l format; uses the modifier class 1; has an opcode of xxxxxxxxxxxx and can generate the xxx flags. This instruction is scalar.}

                \item \textit{\textbf{(LIXOR) Long Immediate bitwise XOR}: this instruction computes \((ra = rb \oplus zeroxt(imm_{26}))\); belongs to the E.l format; uses the modifier class 1; has an opcode of xxxxxxxxxxxx and can generate the xxx flags. This instruction is scalar.}

                \item \textit{\textbf{(LIXNOR) Long Immediate bitwise XNOR}: this instruction computes \((ra = \neg(rb \oplus zeroxt(imm_{26})))\); belongs to the E.l format; uses the modifier class 1; has an opcode of xxxxxxxxxxxx and can generate the xxx flags. This instruction is scalar.}

                \item \textit{\textbf{(IIC) Integer-integer cast}: this isntruction computes \((ra = signxt_{mod}(rb))\); belongs to the C format; uses the modifier class 3; has an opcode of xxxxxxxxxxxx and generates the xxx flags.}

            \end{itemize}

        \end{multicols}

    \subsection{Computational-integer-scalar-advanced}

        \begin{multicols}{2}

            \begin{itemize}

                \item \textit{\textbf{(MUL) }: ...}

                \item \textit{\textbf{(MAC) }: ...}

                \item \textit{\textbf{(DIV) }: ...}
                
                \item \textit{\textbf{(REM) }: ...}

                \item \textit{\textbf{(IMP) }: ...}

                \item \textit{\textbf{(NIMP) }: ...}

                \item \textit{\textbf{(LRT) }: ...}
                
                \item \textit{\textbf{(RRT) }: ...}

                \item \textit{\textbf{(SWP) }: ...}

                \item \textit{\textbf{(IMUL) }: ...}

                \item \textit{\textbf{(IMAC) }: ...}
                
                \item \textit{\textbf{(IDIV) }: ...}

                \item \textit{\textbf{(IREM) }: ...}

                \item \textit{\textbf{(IIMP) }: ...}

                \item \textit{\textbf{(INIMP) }: ...}
                
                \item \textit{\textbf{(IRT) }: ...}

                \item \textit{\textbf{(ISWP) }: ...}

                \item \textit{\textbf{(LIMUL) }: ...}

                \item \textit{\textbf{(LIMAC) }: ...}
                
                \item \textit{\textbf{(LIDIV) }: ...}

                \item \textit{\textbf{(LIREM) }: ...}

                \item \textit{\textbf{(LIIMP) }: ...}

                \item \textit{\textbf{(LINIMP) }: ...}
                
                \item \textit{\textbf{(CTO) }: ...}

                \item \textit{\textbf{(CTLO) }: ...}

                \item \textit{\textbf{(CTTO) }: ...}

                \item \textit{\textbf{(CTZ) }: ...}
                
                \item \textit{\textbf{(CTLZ) }: ...}

                \item \textit{\textbf{(CTTZ) }: ...}

                \item \textit{\textbf{(MWADD) }: ...}

                \item \textit{\textbf{(MWSUB) }: ...}
                
                \item \textit{\textbf{(MWMUL) }: ...}

                \item \textit{\textbf{(MWDIV) }: ...}

                \item \textit{\textbf{(MALSH) }: ...}

                \item \textit{\textbf{(MARSH) }: ...}

                \item \textit{\textbf{(MLLSH) }: ...}
                
                \item \textit{\textbf{(MLRSH) }: ...}

            \end{itemize}  

        \end{multicols}

        [To be continued...]

    \subsection{Data transfer instructions}

        [Coming soon...]

    \subsection{Control transfer instructions}

        [Coming soon...]

    \subsection{System instructions}

        [Coming soon...]