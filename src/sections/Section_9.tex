\section[Instruction list]{\LARGE\underline{Instruction list}}

    \vspace{10pt}

    This section is dedicated to a full and extensive list of all the proposed instructions in the ISA which are divided into their corresponding module. Computational instructions can generate flags but, since there is no flags register, they will be ignored unless the AEXC bit is set for automatic arithmetic exceptions.

    \subsection{Computational-Integer-Scalar-Basic}

        This instruction module provides simple, computational scalar instructions concerning arithmetic and logic:

        \begin{multicols}{2}

            \begin{itemize}

                \item \textit{\textbf{(ADD) Addition}: this instruction computes \((ra = rb + rc)\); belongs to the B format; uses the modifier class 1; has an opcode of xxxxxxxxxxxx and can generate the COVR, OVFL and UNFL flags.}

                \item \textit{\textbf{(SUB) Subtraction}: this instruction computes \((ra = rb - rc)\); belongs to the B format; uses the modifier class 1; has an opcode of xxxxxxxxxxxx and can generate the COVR, OVFL and UNFL flags.}

                \item \textit{\textbf{(AND) Bitwise AND}: this instruction computes \((ra = rb \wedge rc)\); belongs to the B format; uses the modifier class 1; has an opcode of xxxxxxxxxxxx and doesn't generate any flags.}

                \item \textit{\textbf{(NAND) Bitwise NAND}: this instruction computes \((ra = \neg(rb \wedge rc))\); belongs to the B format; uses the modifier class 1; has an opcode of xxxxxxxxxxxx and doesn't generate any flags.}

                \item \textit{\textbf{(OR) Bitwise OR}: this instruction computes \((ra = rb \vee rc)\); belongs to the B format; uses the modifier class 1; has an opcode of xxxxxxxxxxxx and doesn't generate any flags.}

                \item \textit{\textbf{(NOR) Bitwise NOR}: this instruction computes \((ra = \neg(rb \vee rc))\); belongs to the B format; uses the modifier class 1; has an opcode of xxxxxxxxxxxx and doesn't generate any flags.}

                \item \textit{\textbf{(XOR) Bitwise XOR}: this instruction computes \((ra = rb \oplus rc)\); belongs to the B format; uses the modifier class 1; has an opcode of xxxxxxxxxxxx and doesn't generate any flags.}

                \item \textit{\textbf{(XNOR) Bitwise XNOR}: this instruction computes \((ra = \neg(rb \oplus rc))\); belongs to the B format; uses the modifier class 1; has an opcode of xxxxxxxxxxxx and doesn't generate any flags.}

                \item \textit{\textbf{(ALSH) Arithmetic Left Shift}: this instruction computes \((ra = rb <<< rc)\); belongs to the B format; uses the modifier class 1; has an opcode of xxxxxxxxxxxx and can generate the xxx flags.}

                \item \textit{\textbf{(ARSH) Arithmetic Right Shift}: this instruction computes \((ra = rb >>> rc)\); belongs to the E format; uses the modifier class 1; has an opcode of xxxxxxxxxxxx and can generate the xxx flags.}

                \item \textit{\textbf{(LLSH) Logical Left Shift}: this instruction computes \((ra = rb << rc)\); belongs to the E format; uses the modifier class 1; has an opcode of xxxxxxxxxxxx and can generate the xxx flags.}

                \item \textit{\textbf{(LRSH) Logical Right Shift}: this instruction computes \((ra = rb >> rc)\); belongs to the E format; uses the modifier class 1; has an opcode of xxxxxxxxxxxx and can generate the xxx flags.}

                \item \textit{\textbf{(IADD) Immediate Addition}: this instruction computes \((ra = rb + signxt(imm_8))\); belongs to the E format; uses the modifier class 1; has an opcode of xxxxxxxxxxxx and can generate the xxx flags.}

                \item \textit{\textbf{(ISUB) Immediate Subtraction}: this instruction computes \((ra = rb - signxt(imm_8))\); belongs to the E format; uses the modifier class 1; has an opcode of xxxxxxxxxxxx and can generate the xxx flags.}

                \item \textit{\textbf{(IAND) Immediate Bitwise AND}: this instruction computes \((ra = rb \wedge zeroxt(imm_8))\); belongs to the E format; uses the modifier class 1; has an opcode of xxxxxxxxxxxx and can generate the xxx flags.}

                \item \textit{\textbf{(INAND) Immediate Bitwise NAND}: this instruction computes \((ra = \neg(rb \wedge zeroxt(imm_8)))\); belongs to the E format; uses the modifier class 1; has an opcode of xxxxxxxxxxxx and can generate the xxx flags.}

                \item \textit{\textbf{(IOR) Immediate Bitwise OR}: this instruction computes \((ra = rb \vee zeroxt(imm_8))\); belongs to the E format; uses the modifier class 1; has an opcode of xxxxxxxxxxxx and can generate the xxx flags.}

                \item \textit{\textbf{(INOR) Immediate Bitwise NOR}: this instruction computes \((ra = \neg(rb \vee zeroxt(imm_8)))\); belongs to the E format; uses the modifier class 1; has an opcode of xxxxxxxxxxxx and can generate the xxx flags.}

                \item \textit{\textbf{(IXOR) Immediate Bitwise XOR}: this instruction computes \((ra = rb \oplus zeroxt(imm_8))\); belongs to the E format; uses the modifier class 1; has an opcode of xxxxxxxxxxxx and can generate the xxx flags.}

                \item \textit{\textbf{(IXNOR) Immediate Bitwise XNOR}: this instruction computes \((ra = \neg(rb \oplus zeroxt(imm_8)))\); belongs to the E format; uses the modifier class 1; has an opcode of xxxxxxxxxxxx and can generate the xxx flags.}

                \item \textit{\textbf{(IASH) Immediate Arithmetic Shift}: this instruction computes \((ra = rb <<<>>> imm_8)\); belongs to the E format; uses the modifier class 1; has an opcode of xxxxxxxxxxxx and can generate the xxx flags. The direction of the shift is dictated by imm[7] bit.}

                \item \textit{\textbf{(ILSH) Immediate Logical Shift}: this instruction computes \((ra = rb <<>> imm_8)\); belongs to the E format; uses the modifier class 1; has an opcode of xxxxxxxxxxxx and can generate the xxx flags. The direction of the shift is dictated by imm[7] bit.}

                \item \textit{\textbf{(LIADD) Long immediate addition}: this instruction computes \((ra = rb + signxt(imm_{26}))\); belongs to the E.l format; uses the modifier class 1; has an opcode of xxxxxxxxxxxx and can generate the xxx flags.}

                \item \textit{\textbf{(LISUB) Long immediate subtraction}: this instruction computes \((ra = rb - signxt(imm_{26}))\); belongs to the E.l format; uses the modifier class 1; has an opcode of xxxxxxxxxxxx and can generate the xxx flags.}

                \item \textit{\textbf{(LIAND) Long Immediate bitwise AND}: this instruction computes \((ra = rb \wedge zeroxt(imm_{26}))\); belongs to the E format; uses the modifier class 1; has an opcode of xxxxxxxxxxxx and can generate the xxx flags.}

                \item \textit{\textbf{(LINAND) Long Immediate bitwise NAND}: this instruction computes \((ra = \neg(rb \wedge zeroxt(imm_{26})))\); belongs to the E.l format; uses the modifier class 1; has an opcode of xxxxxxxxxxxx and can generate the xxx flags.}

                \item \textit{\textbf{(LIOR) Long Immediate bitwise OR}: this instruction computes \((ra = rb \vee zeroxt(imm_{26}))\); belongs to the E.l format; uses the modifier class 1; has an opcode of xxxxxxxxxxxx and can generate the xxx flags.}

                \item \textit{\textbf{(LINOR) Long Immediate bitwise NOR}: this instruction computes \((ra = \neg(rb \vee zeroxt(imm_{26})))\); belongs to the E.l format; uses the modifier class 1; has an opcode of xxxxxxxxxxxx and can generate the xxx flags.}

                \item \textit{\textbf{(LIXOR) Long Immediate bitwise XOR}: this instruction computes \((ra = rb \oplus zeroxt(imm_{26}))\); belongs to the E.l format; uses the modifier class 1; has an opcode of xxxxxxxxxxxx and can generate the xxx flags.}

                \item \textit{\textbf{(LIXNOR) Long Immediate bitwise XNOR}: this instruction computes \((ra = \neg(rb \oplus zeroxt(imm_{26})))\); belongs to the E.l format; uses the modifier class 1; has an opcode of xxxxxxxxxxxx and can generate the xxx flags.}

                \item \textit{\textbf{(IIC) Integer-Integer Cast}: this instruction computes \((ra = signxt_{mod}(rb))\); belongs to the C format; uses the modifier class 3; has an opcode of xxxxxxxxxxxx and generates the xxx flags.}

            \end{itemize}

        \end{multicols}

    \par\noindent\rule{\textwidth}{0.4pt}
    \textit{This instruction module adds simple instructions such as addition, subtraction, logic and some bit-shifting. Immediate variants can choose between two immediate lengths at the cost of a longer instruction. The IIC instructions allows the programmer to change data type length and the bit-field instructions allows precise manipulation of individual bits, including the loading of immediate values directly.}
    \par\noindent\rule{\textwidth}{0.4pt}

    \subsection{Computational-integer-scalar-advanced}

        \begin{multicols}{2}

            \begin{itemize}

                \item \textit{\textbf{(MUL) Multiplication}: ...}

                \item \textit{\textbf{(MAC) Multiply and Accumulate}: ...}

                \item \textit{\textbf{(DIV) Division}: ...}
                
                \item \textit{\textbf{(REM) Remainder}: ...}

                \item \textit{\textbf{(IMP) Bitwise IMP}: ...}

                \item \textit{\textbf{(NIMP) Bitwise NIMP}: ...}

                \item \textit{\textbf{(LRT) Left Rotation}: ...}
                
                \item \textit{\textbf{(RRT) Right Rotation}: ...}

                \item \textit{\textbf{(SWP) Bit Swap}: ...}

                \item \textit{\textbf{(IMUL) Immediate Multiplication}: ...}

                \item \textit{\textbf{(IMAC) Immediate Multiply and Accumulate}: ...}
                
                \item \textit{\textbf{(IDIV) Immediate Division}: ...}

                \item \textit{\textbf{(IREM) Immediate Remainder}: ...}

                \item \textit{\textbf{(IIMP) Immediate Bitwise IMP}: ...}

                \item \textit{\textbf{(INIMP) Immediate Bitwise NIMP}: ...}
                
                \item \textit{\textbf{(IRT) Immediate Rotation}: ...}

                \item \textit{\textbf{(ISWP) Immediate Bit Swap}: ...}

                \item \textit{\textbf{(LIMUL) Long Immediate Multiplication}: ...}

                \item \textit{\textbf{(LIMAC) Long Immediate Multiply and Accumulate}: ...}
                
                \item \textit{\textbf{(LIDIV) Long Immediate Division}: ...}

                \item \textit{\textbf{(LIREM) Long Immediate Remainder}: ...}

                \item \textit{\textbf{(LIIMP) Long Immediate Bitwise IMP}: ...}

                \item \textit{\textbf{(LINIMP) Long Immediate Bitwise NIMP}: ...}
                
                \item \textit{\textbf{(CTO) Count Ones}: ...}

                \item \textit{\textbf{(CTLO) Count Leading Ones}: ...}

                \item \textit{\textbf{(CTTO) Count Trailing Ones}: ...}

                \item \textit{\textbf{(CTZ) Count Zeros}: ...}
                
                \item \textit{\textbf{(CTLZ) Count Leading Zeros}: ...}

                \item \textit{\textbf{(CTTZ) Count Trailing Zeros}: ...}

                \item \textit{\textbf{(MWADD) }: ...}

                \item \textit{\textbf{(MWSUB) }: ...}
                
                \item \textit{\textbf{(MWMUL) }: ...}

                \item \textit{\textbf{(MWDIV) }: ...}

                \item \textit{\textbf{(MALSH) }: ...}

                \item \textit{\textbf{(MARSH) }: ...}

                \item \textit{\textbf{(MLLSH) }: ...}
                
                \item \textit{\textbf{(MLRSH) }: ...}

            \end{itemize}  

        \end{multicols}

    \par\noindent\rule{\textwidth}{0.4pt}
    \textit{This instruction module adds more advanced and complex instructions such as multiplication, division, remainder, rotates and multi-word specific operations. Instructions concerning multiplication will only save the lower half of the result in order to streamline simpler implementations. Multi-word instructions can help with arbitrary precision arithmetic and are implemented without the need of a dedicated ``flag register'' at the cost of more operands. The low average dynamic count of these instructions, though, shouldn't result in notable structural hazards even if the number of register read ports is not high.}
    \par\noindent\rule{\textwidth}{0.4pt}

    \subsection{Computational-integer-vector-basic}

        This instruction modules requires the presence of the vector registers (VGPRs), the VMSK register located in the special purpose bank and the VLEN bits in the CSR register. The hardware must also provide the ability to execute instructions of the CISB module in vector mode with the help of the 'vv' bits in the modifier classes. The vector modes must be capable of executing \textit{vector-vector} and \textit{vector-scalar} configurations. This module doesn't explicitly add extra instructions because it essentially extends the capabilities of the CISB module.

    \par\noindent\rule{\textwidth}{0.4pt}
    \textit{...}
    \par\noindent\rule{\textwidth}{0.4pt}

    \subsection{Computational-integer-vector-advanced}

        This instruction modules requires the presence of the vector registers (VGPRs), the VMSK register located in the special purpose bank and the VLEN bits in the CSR register. The hardware must also provide the ability to execute instructions of the CISA module in vector mode with the help of the 'vv' bits in the modifier classes. The vector modes must be capable of executing \textit{vector-vector} and \textit{vector-scalar} configurations. This module doesn't explicitly add extra instructions because it essentially extends the capabilities of the CISA module.

    \par\noindent\rule{\textwidth}{0.4pt}
    \textit{...}
    \par\noindent\rule{\textwidth}{0.4pt}

    \subsection{Computational-integer-vector-reduction}

        This instruction modules requires the presence of the vector registers (VGPRs), the VMSK register located in the special purpose bank and the VLEN bits in the CSR register. This module also adds xxx extra instructions that are vector only. The 'd' bit for the modifier class 2 of the C format only affects non commutative operations, otherwise it is ignored. The proposed instructions are:

        \begin{multicols}{2}

            \begin{itemize}

                \item \textit{\textbf{(RADD) Reduction Addition}: ...}

                \item \textit{\textbf{(RSUB) Reduction Subtraction}: ...}

                \item \textit{\textbf{(RMUL) Reduction Multiplication}: ...}

                \item \textit{\textbf{(RMAC) Reduction Multiply and Accumulate}: ...}

                \item \textit{\textbf{(RDIV) Reduction Division}: ...}

                \item \textit{\textbf{(RAND) Reduction Bitwise AND}: ...}

                \item \textit{\textbf{(RNAND) Reduction Bitwise NAND}: ...}

                \item \textit{\textbf{(ROR) Reduction Bitwise OR}: ...}

                \item \textit{\textbf{(RNOR) Reduction Bitwise NOR}: ...}

                \item \textit{\textbf{(RXOR) Reduction Bitwise XOR}: ...}

                \item \textit{\textbf{(RXNOR) Reduction Bitwise XNOR}: ...}

                \item \textit{\textbf{(RIMP) Reduction Bitwise IMP}: ...}

                \item \textit{\textbf{(RNIMP) Reduction Bitwise NIMP}: ...}

            \end{itemize}

        \end{multicols}

        [To be continued...]

    \subsection{Data transfer-scalar-basic}

        [Coming soon...]

    \subsection{Control transfer instructions}

        [Coming soon...]

    \subsection{System instructions}

        [Coming soon...]