\section[ISA modules list]{\LARGE\underline{ISA modules list}}

    \vspace{10pt}

    This section is dedicated to provide a brief but full description of all the different modules that the FabRISC ISA offers. There are no mandatory modules in this specification in order to maximize the flexibility, however, once a particular extension is chosen, the hardware must provide all the features and abstractions of said extension. The requirements for each and every module will be extensively explained in the next sections. The following, is the list of all modules:

    \begin{enumerate}

        \item \textit{\textbf{Computational}: these modules include instructions that perform arithmetic or logic operations:}

            \begin{itemize}

                \item 1) \textit{\textbf{CISB}: computational-integer-scalar-basic.}
                \item 2) \textit{\textbf{CISA}: computational-integer-scalar-advanced.}

                \item 3) \textit{\textbf{CIVB}: computational-integer-vector-basic.}
                \item 4) \textit{\textbf{CIVA}: computational-integer-vector-advanced.}
                \item 5) \textit{\textbf{CIVR}: computational-integer-vector-reduction.}

                \item 6) \textit{\textbf{CFSB}: computational-FP-scalar-basic.}
                \item 7) \textit{\textbf{CFSA}: computational-FP-scalar-advanced.}

                \item 8) \textit{\textbf{CFVB}: computational-FP-vector-basic.}
                \item 9) \textit{\textbf{CFVA}: computational-FP-vector-advanced.}
                \item 10) \textit{\textbf{CFVR}: computational-FP-vector-reduction.}

                \item 11) \textit{\textbf{CCIB}: computational-compressed-integer-basic.}
                \item 12) \textit{\textbf{CCIA}: computational-compressed-integer-advanced.}

                \item 13) \textit{\textbf{CCFB}: computational-compressed-FP-basic.}
                \item 14) \textit{\textbf{CCFA}: computational-compressed-FP-advanced.}

                \item 15) \textit{\textbf{CSM}: computational-scalar-mask.}

            \end{itemize}

        \item \textit{\textbf{Data transfer}: these modules include instructions that perform memory load, store, register move and swap operations:}

            \begin{itemize}

                \item 16) \textit{\textbf{DSB}: data transfer-scalar-basic.}
                \item 17) \textit{\textbf{DSA}: data transfer-scalar-advanced.}

                \item 18) \textit{\textbf{DVB}: data transfer-vector-basic.}
                \item 19) \textit{\textbf{DVA}: data transfer-vector-advanced.}
                \item 20) \textit{\textbf{DVG}: data transfer-vector-gather scatter.}

                \item 21) \textit{\textbf{DC}: data transfer-compressed.}
                \item 22) \textit{\textbf{DB}: data transfer-block.}

            \end{itemize}

        \item \textit{\textbf{Flow transfer}: these modules include instructions that perform conditional branch, jump, call and return operations:}

            \begin{itemize}

                \item 23) \textit{\textbf{FISB}: flow transfer-integer-scalar-basic.}
                \item 24) \textit{\textbf{FISA}: flow transfer-integer-scalar-advanced.}
                \item 25) \textit{\textbf{FISM}: flow transfer-integer-scalar-mask.}

                \item 26) \textit{\textbf{FIV}: flow transfer-integer-vector.}

                \item 27) \textit{\textbf{FFSB}: flow transfer-FP-scalar-basic.}
                \item 28) \textit{\textbf{FFSA}: flow transfer-FP-scalar-advanced.}
                \item 29) \textit{\textbf{FFSM}: flow transfer-FP-scalar-mask.}

                \item 30) \textit{\textbf{FFVB}: flow transfer-FP-vector.}

                \item 31) \textit{\textbf{FC}: flow transfer-compressed.}

            \end{itemize}

        \item \textit{\textbf{System}: these modules include instructions that perform system various related operations:}

            \begin{itemize}

                \item 32) \textit{\textbf{SB}: system-basic.}
                \item 33) \textit{\textbf{SA}: system-advanced.}

            \end{itemize}

        \item \textit{\textbf{Miscellaneous}: these modules include miscellaneous components such as extra instructions, register banks, events and more:}

        \item 34) \textit{\textbf{FNC}: fences.}
        \item 35) \textit{\textbf{TM}: transactional-memory.}
        \item 36) \textit{\textbf{AM}: atomic-memory.}

        \item 37) \textit{\textbf{EXC}: exceptions.}
        \item 38) \textit{\textbf{FLT}: faults.}
        \item 39) \textit{\textbf{IOINT}: IO interrupts.}           
        \item 40) \textit{\textbf{IPCINT}: IPC interrupts.}

        \item 41) \textit{\textbf{DBGR}: debugging registers.}
        \item 42) \textit{\textbf{PERFC}: performance counters.}

        \item 43) \textit{\textbf{SUPER}: supervisor.}

        \end{enumerate}

    \par\noindent\rule{\textwidth}{0.4pt}
    \textit{The high number of modules and extensions is very useful because it allows the hardware designers to only implement what they really want and very little extra. The fact that there is no explicit mandatory subset of the ISA helps can with specialized systems. With this, it becomes perfectly possible to create, for example, a floating-point only processor with no integer instructions to alleviate overheads. This method makes possible the implementation of silly things such as having no flow transfer instructions, thus it kind of relies on the common sense of the hardware designers when it comes to realize sensible microarchitectures.}
    \par\noindent\rule{\textwidth}{0.4pt}
