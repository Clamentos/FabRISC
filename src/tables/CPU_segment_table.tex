\begin{table}[hbt!] % fully done

    \begin{center}

    \begin{tabular}{|c|l|r|l|}

        \hline
        Address & Name & Size & Category\\
        \hline
        \addlinespace[10pt]
        \hline
        00 & CPU ID & 8 & CPU info\\
        \hline
        08 & Implemented ISA modules & 8 & CPU info\\
        \hline
        10 & Number of cores & 1 & CPU info\\
        \hline
        11 & SMT degree & 1 & CPU info\\
        \hline
        12 & WLEN and MXVL & 1 & CPU info\\
        \hline
        13 & CPU clock speed & 4 & CPU info\\
        \hline
        17 & CPU temperature & 2 & CPU info\\
        \hline
        19 & CPU voltage & 2 & CPU info\\
        \hline
        1B & TLB information & 32 & CPU info\\
        \hline
        3B & TLB hierarchy & 4 & CPU info\\
        \hline
        3F & Cache information & 32 & CPU info\\
        \hline
        5F & Cache hierarchy & 4 & CPU info\\
        \hline
        63 & IO channels & 1 & IO info\\
        \hline
        64 & IO clock speed & 4 & IO info\\
        \hline
        68 & IO voltage & 2 & IO info\\
        \hline
        6A & Available memory & 8 & MEM info\\
        \hline
        72 & Memory channels & 1 & MEM info\\
        \hline
        73 & CAS latency & 1 & MEM info\\
        \hline
        74 & RAS to CAS latency & 1 & MEM info\\
        \hline
        75 & RAS precharge latency & 1 & MEM info\\
        \hline
        76 & CMD latency & 1 & MEM info\\
        \hline
        77 & Memory clock speed & 4 & MEM info\\
        \hline
        7B & Memory temperature & 2 & MEM info\\
        \hline
        7D & Memory voltage & 2 & MEM info\\
        \hline
        7F & Reserved & 1 & Reserved\\
        \hline

    \end{tabular}

    \caption[CPU segment table]{CPU segment table.}

    \end{center}

\end{table}