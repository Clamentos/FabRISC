\begin{table}[hbt!]

    \begin{center}

    \begin{tabular}{|c|l|r|l|}

        \hline
        Address & Name & Size & Category\\
        \hline
        \addlinespace[10pt]
        \hline
        00 & CPU ID & 8 & CPU info\\
        \hline
        07 & Implemented ISA modules & 4 & CPU info\\
        \hline
        0B & Number of cores & 1 & CPU info\\
        \hline
        0C & SMT degree & 1 & CPU info\\
        \hline
        0D & WLEN and MXVL & 1 & CPU info\\
        \hline
        0E & CPU clock speed & 4 & CPU info\\
        \hline
        11 & CPU temperature & 2 & CPU info\\
        \hline
        13 & CPU voltage & 2 & CPU info\\
        \hline
        15 & TLB information & 32 & CPU info\\
        \hline
        35 & TLB hierarchy & 4 & CPU info\\
        \hline
        38 & Cache information & 32 & CPU info\\
        \hline
        58 & Cache hierarchy & 4 & CPU info\\
        \hline
        5B & IO channels & 1 & IO info\\
        \hline
        5C & IO clock speed & 4 & IO info\\
        \hline
        60 & IO voltage & 2 & IO info\\
        \hline
        62 & Available memory & 8 & MEM info\\
        \hline
        6A & Memory channels & 1 & MEM info\\
        \hline
        6B & CAS latency & 1 & MEM info\\
        \hline
        6C & RAS to CAS latency & 1 & MEM info\\
        \hline
        6D & RAS precharge latency & 1 & MEM info\\
        \hline
        6E & CMD latency & 1 & MEM info\\
        \hline
        6F & Memory clock speed & 4 & MEM info\\
        \hline
        73 & Memory temperature & 2 & MEM info\\
        \hline
        75 & Memory voltage & 2 & MEM info\\
        \hline
        77 & Reserved & 9 & Reserved\\
        \hline

    \end{tabular}

    \caption[CPU segment table]{CPU segment table.}

    \end{center}

\end{table}