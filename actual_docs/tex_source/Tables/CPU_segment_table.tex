\begin{table}[hbt!] % ok

    \begin{center}

    \begin{tabular}{|c|c|c|c|}

        \hline
        Address & Name & Size & Category\\
        \hline
        0 & CPU ID & 8 & CPU info\\
        \hline
        7 & Implemented ISA modules & 4 & CPU info\\
        \hline
        11 & Number of cores & 1 & CPU info\\
        \hline
        12 & SMT degree & 1 & CPU info\\
        \hline
        13 & WLEN and MXVL & 1 & CPU info\\
        \hline
        14 & CPU clock speed & 4 & CPU info\\
        \hline
        17 & CPU temperature & 2 & CPU info\\
        \hline
        19 & CPU voltage & 2 & CPU info\\
        \hline
        21 & TLB information & 32 & CPU info\\
        \hline
        53 & TLB hierarchy & 4 & CPU info\\
        \hline
        56 & Cache information & 32 & CPU info\\
        \hline
        88 & Cache hierarchy & 4 & CPU info\\
        \hline
        91 & IO channels & 1 & IO info\\
        \hline
        92 & IO clock speed & 4 & IO info\\
        \hline
        96 & IO voltage & 2 & IO info\\
        \hline
        98 & Available memory & 8 & MEM info\\
        \hline
        106 & Memory channels & 1 & MEM info\\
        \hline
        107 & CAS latency & 1 & MEM info\\
        \hline
        108 & RAS to CAS latency & 1 & MEM info\\
        \hline
        109 & RAS precharge latency & 1 & MEM info\\
        \hline
        110 & CMD latency & 1 & MEM info\\
        \hline
        111 & Memory clock speed & 4 & MEM info\\
        \hline
        115 & Memory temperature & 2 & MEM info\\
        \hline
        117 & Memory voltage & 2 & MEM info\\
        \hline
        119 & Reserved & 9 & Reserved\\
        \hline

    \end{tabular}

    \caption[CPU segment table]{CPU segment table.}

    \end{center}

\end{table}